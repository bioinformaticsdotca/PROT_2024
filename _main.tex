% Options for packages loaded elsewhere
\PassOptionsToPackage{unicode}{hyperref}
\PassOptionsToPackage{hyphens}{url}
%
\documentclass[
]{book}
\usepackage{amsmath,amssymb}
\usepackage{iftex}
\ifPDFTeX
  \usepackage[T1]{fontenc}
  \usepackage[utf8]{inputenc}
  \usepackage{textcomp} % provide euro and other symbols
\else % if luatex or xetex
  \usepackage{unicode-math} % this also loads fontspec
  \defaultfontfeatures{Scale=MatchLowercase}
  \defaultfontfeatures[\rmfamily]{Ligatures=TeX,Scale=1}
\fi
\usepackage{lmodern}
\ifPDFTeX\else
  % xetex/luatex font selection
\fi
% Use upquote if available, for straight quotes in verbatim environments
\IfFileExists{upquote.sty}{\usepackage{upquote}}{}
\IfFileExists{microtype.sty}{% use microtype if available
  \usepackage[]{microtype}
  \UseMicrotypeSet[protrusion]{basicmath} % disable protrusion for tt fonts
}{}
\makeatletter
\@ifundefined{KOMAClassName}{% if non-KOMA class
  \IfFileExists{parskip.sty}{%
    \usepackage{parskip}
  }{% else
    \setlength{\parindent}{0pt}
    \setlength{\parskip}{6pt plus 2pt minus 1pt}}
}{% if KOMA class
  \KOMAoptions{parskip=half}}
\makeatother
\usepackage{xcolor}
\usepackage{longtable,booktabs,array}
\usepackage{calc} % for calculating minipage widths
% Correct order of tables after \paragraph or \subparagraph
\usepackage{etoolbox}
\makeatletter
\patchcmd\longtable{\par}{\if@noskipsec\mbox{}\fi\par}{}{}
\makeatother
% Allow footnotes in longtable head/foot
\IfFileExists{footnotehyper.sty}{\usepackage{footnotehyper}}{\usepackage{footnote}}
\makesavenoteenv{longtable}
\usepackage{graphicx}
\makeatletter
\def\maxwidth{\ifdim\Gin@nat@width>\linewidth\linewidth\else\Gin@nat@width\fi}
\def\maxheight{\ifdim\Gin@nat@height>\textheight\textheight\else\Gin@nat@height\fi}
\makeatother
% Scale images if necessary, so that they will not overflow the page
% margins by default, and it is still possible to overwrite the defaults
% using explicit options in \includegraphics[width, height, ...]{}
\setkeys{Gin}{width=\maxwidth,height=\maxheight,keepaspectratio}
% Set default figure placement to htbp
\makeatletter
\def\fps@figure{htbp}
\makeatother
\setlength{\emergencystretch}{3em} % prevent overfull lines
\providecommand{\tightlist}{%
  \setlength{\itemsep}{0pt}\setlength{\parskip}{0pt}}
\setcounter{secnumdepth}{5}
\usepackage{booktabs}

\usepackage{color}
\usepackage{framed}
\setlength{\fboxsep}{.8em}

% These colours were manually entered, they shouldn't matter unless you want pdf output

\newenvironment{redbox}{
  \definecolor{shadecolor}{RGB}{243, 154, 157}
  \color{white}
  \begin{shaded}}
 {\end{shaded}}

\newenvironment{bluebox}{
  \definecolor{shadecolor}{RGB}{172, 210, 237}
  \color{white}
  \begin{shaded}}
 {\end{shaded}}

\newenvironment{greenbox}{
  \definecolor{shadecolor}{RGB}{141, 181, 128}
  \color{white}
  \begin{shaded}}
 {\end{shaded}}
\ifLuaTeX
  \usepackage{selnolig}  % disable illegal ligatures
\fi
\usepackage[]{natbib}
\bibliographystyle{plainnat}
\usepackage{bookmark}
\IfFileExists{xurl.sty}{\usepackage{xurl}}{} % add URL line breaks if available
\urlstyle{same}
\hypersetup{
  pdftitle={Proteomics 2024},
  pdfauthor={Faculty: Jennifer Geddes-McAlister, Florence Roux-Dalvai, Ben Muselius, Marie Pier Scott-Boyer, Mickael Leclercq},
  hidelinks,
  pdfcreator={LaTeX via pandoc}}

\title{Proteomics 2024}
\author{Faculty: Jennifer Geddes-McAlister, Florence Roux-Dalvai, Ben Muselius, Marie Pier Scott-Boyer, Mickael Leclercq}
\date{September 16, 2024 - September 18, 2024}

\begin{document}
\maketitle

{
\setcounter{tocdepth}{1}
\tableofcontents
}
\part{Introduction}\label{part-introduction}

\chapter{Workshop Info}\label{workshop-info}

Welcome to the 2024 Proteomics Canadian Bioinformatics Workshop webpage!

\section{Class Photo}\label{class-photo}

\section{Schedule}\label{schedule}

\begin{longtable}[]{@{}
  >{\centering\arraybackslash}p{(\columnwidth - 10\tabcolsep) * \real{0.0437}}
  >{\centering\arraybackslash}p{(\columnwidth - 10\tabcolsep) * \real{0.1503}}
  >{\centering\arraybackslash}p{(\columnwidth - 10\tabcolsep) * \real{0.0437}}
  >{\centering\arraybackslash}p{(\columnwidth - 10\tabcolsep) * \real{0.5874}}
  >{\centering\arraybackslash}p{(\columnwidth - 10\tabcolsep) * \real{0.0437}}
  >{\centering\arraybackslash}p{(\columnwidth - 10\tabcolsep) * \real{0.1311}}@{}}
\toprule\noalign{}
\begin{minipage}[b]{\linewidth}\centering
\textbf{Time (EST)}
\end{minipage} & \begin{minipage}[b]{\linewidth}\centering
\textbf{Day 1: Mon, Sept.~16}
\end{minipage} & \begin{minipage}[b]{\linewidth}\centering
\textbf{Time (EST)}
\end{minipage} & \begin{minipage}[b]{\linewidth}\centering
\textbf{Day 2: Tues, Sept.~17}
\end{minipage} & \begin{minipage}[b]{\linewidth}\centering
\textbf{Time (EST)}
\end{minipage} & \begin{minipage}[b]{\linewidth}\centering
\textbf{Day 3: Wed, Sept.~18}
\end{minipage} \\
\midrule\noalign{}
\endhead
\bottomrule\noalign{}
\endlastfoot
8:30 & Arrivals \& Check-in & 8:30 & Arrivals & 8:30 & Arrivals \\
9:00 & Welcome (Nia Hughes) & 9:00 & Module 4 (Droit) & 9:00 & Module 4\&5 - Lab (Droit \& Geddes-McAlister) \\
9:15 & Module 1 (Geddes-McAlister) & 9:45 & Module 4 - Lab (Droit) & 10:30 & Break (30min) \\
10:45 & Break (30min) (sponsored by Moms in Proteomics) & 10:45 & Break (30min) & 11:00 & Module 4\&5 - Lab (Droit \& Geddes-McAlister) \\
11:15 & Module 2 (Droit) & 11:15 & Module 4 - Lab (Droit) & 12:30 & Lunch (1h) \\
12:30 & Lunch (1h) + Facility tour & 12:30 & Lunch (1h) & 13:30 & Module 6 (Geddes-McAlister) + Group discussion \\
13:30 & Module 3 (Geddes-McAlister) & 13:30 & Module 5 (Geddes-McAlister) & 15:00 & Break (30min) \\
14:30 & Module 3 Lab (Geddes-McAlister) & 15:00 & Break (30min) & 15:30 & Module 7 (Droit) \\
15:30 & Break (30min) & 15:30 & Module 5 (Geddes-McAlister) & 16:30 & Survey \& Closing Remarks \\
16:00 & Module 3 Lab (Geddes-McAlister) & 16:15 & Module 5 Lab (Geddes-McAlister) & 17:00 & Finished \\
17:30 & Networking \& dinner (Sponsored by CNPN) + Group photo & 17:00 & Guest lecture: New technology and application talk by Dr.~Kevin Cormier, Proteomics Scientist, Protein Chemistry Section/Integrated Data Science Section, Research Technologies Branch, National Institutes of Health & & \\
\end{longtable}

\section{Pre-work}\label{pre-work}

You can find your pre-work \href{https://docs.google.com/forms/d/e/1FAIpQLScSkUfju24IarDfunnvCKcBvN8SW7-m-5arH-zlfrtY0ahsFw/viewform}{here.}

\chapter{Data and Compute Setup}\label{data-and-compute-setup}

\subsubsection{Course data downloads}\label{course-data-downloads}

\href{https://drive.google.com/drive/folders/1kSsR2d_cI5FeDFW5lCD7In_Z5-KHLrBt?usp=drive_link}{DDA}
\href{https://drive.google.com/drive/folders/18uY7-vYLHd4TbgJ54Snay3MvSWH4LOOY?usp=drive_link}{DIA-NN}
\href{https://drive.google.com/drive/folders/1gweCDB6GbBH_6Gw1X4RPNQe3_H2nsmcY?usp=sharing}{Module 5 Lab (R)}

\subsubsection{Compute setup}\label{compute-setup}

\href{https://bioinformaticsdotca.github.io/AWS_setup_Windows}{Log into your Windows server at AWS}

\part{Modules}\label{part-modules}

\chapter{Module 1: Introduction to Mass Spectrometry and Proteomics}\label{module-1-introduction-to-mass-spectrometry-and-proteomics}

\section{Lecture}\label{lecture}

\chapter{Module 2: Experimental Design and Analytical Strategies}\label{module-2-experimental-design-and-analytical-strategies}

\section{Lecture}\label{lecture-1}

\chapter{Module 3: Introduction to Software Platforms}\label{module-3-introduction-to-software-platforms}

\section{Lecture}\label{lecture-2}

\chapter{Module 4: Data-Independent Acquisition (DIA) Analysis}\label{module-4-data-independent-acquisition-dia-analysis}

\section{Lecture}\label{lecture-3}

\section{Lab}\label{lab}

\chapter{Module 5: Best Practices in Data Analysis, Statistics, and Visualization}\label{module-5-best-practices-in-data-analysis-statistics-and-visualization}

\section{Lecture}\label{lecture-4}

\section{Lab}\label{lab-1}

\section{Cheat Sheets}\label{cheat-sheets}

\chapter{Module 6: Considerations and Tools for Complex Datasets}\label{module-6-considerations-and-tools-for-complex-datasets}

\section{Lecture}\label{lecture-5}

\chapter{Module 7: Multi-omics Integration \& Current and Future Advances of AI in Proteomics}\label{module-7-multi-omics-integration-current-and-future-advances-of-ai-in-proteomics}

\section{Lecture}\label{lecture-6}

  \bibliography{book.bib,packages.bib}

\end{document}
